\documentclass[10pt]{examdesign}

% loads all necessary packages (feel free to customize for your needs)

\usepackage{amsmath}
\usepackage{graphicx}
\usepackage[utf8]{inputenc}
\usepackage{pifont}
\usepackage[english,brazil]{babel}
\usepackage{fancyhdr}
\pagestyle{fancy}
\usepackage{totpages}
\usepackage[margin=0.9in]{geometry}
\renewcommand{\headrulewidth}{0.0pt}
\usepackage{framed} 

\SectionFont{\large\sffamily}
\Fullpages
\ContinuousNumbering
\DefineAnswerWrapper{}{}
\NumberOfVersions{1}

%\ShortKey		% shows a short key answer sheet
\NoKey			% Print key ?

\NoRearrange 	% Random arrange with latex? (leave it UNCOMMENTED, R will do all of the dirty work)

\newcommand{\myversion}{} % this line is very important as it locates and allows the printed test to show the number of  the test (DONT DELETE IT)

\rfoot{Random test \# \myversion}

\begin{document}

\begin{examtop}
	
\begin{center}
    \textbf{\Large  Name of my University} \\
    \textbf{\Large  My Department} \vspace{0.5cm}  \\   
    \textbf{\Large  Name of Class} \\
	\textbf{\Large  Author: My name} \\
    \textbf{\Large Exam Name: Version \myversion } \\ 
    \textbf{\Large Current Term}
 \end{center}

\vspace{1cm}
Name: \rule{4in}{.4pt}  \quad  \noindent Card:\enspace\hrulefill
 
%\rule[1ex]{\textwidth}{2pt}

\vspace{1cm}

\small

\begin{framed}

Instructions for this test:

\begin{itemize}
	
	\item Make sure you finish before the end of time
	
	\item There is only one correct answer in all multiple choice questions

	\item Don't cheat!
	
\end{itemize}

\vspace{0.5cm}

{\large \emph{Good Luck!}}
\end{framed}

\begin{framed}
	Instructions on how to use MyRandomTest (DELETE THIS in FINAL VERSION)
	\begin{itemize}
		
		\item The switch statements for the text of the questions, including main text and answers are defined as @{Text for version 1}|{Text for version 2}@.  Fell free to define as many versions as needed.
		
		\item You can define the right answer in each version of the test by assigning the character [ver] in the right answer for version ver. See the rest of the file for examples. 
		
		
	\end{itemize}

\end{framed}

\end{examtop}


\vspace{1cm}

\begin{multiplechoice}[resetcounter=no,  examcolumns=1]

\begin{question}
	
	Given the next five options, which on is the correct answer?
	
	\choice{Choice 1}
	
	\choice{Choice 2}
	
	\choice{Choice 3}
	
	\choice{Choice 4}
	
	\choice[!]{Choice 5 - CORRECT}
	
\end{question}

\begin{question}
	
	Consider the following statements:
	
	\begin{enumerate}[I]
		\item The color of the sky is generally @{blue}|{red}@
		
		\item R has a @{high}|{low}@ number of packages for empirical research in Finance
		
		\item Microsoft word is much @{better}|{worse}@ than latex for handling structured documents
	\end{enumerate}
	
	Are correct:

	\choice{[1] I and II}
	
	\choice{II and III}
	
	\choice{I, II and III}
	
	\choice{[2] Only II}
	
	\choice{Only I}
	
\end{question}


\begin{question}
	
	Which one of the next questions do you think is the correct one? (there are three version of this question)
	
	\choice{Choice 1 - Incorrect in all versions}
	
	\choice{[1]Choice 2 - @{Correct in version 1}|{Incorrect in version 2}|{Incorrect in version 3}@}
	
	\choice{[3] Choice 3 - @{Incorrect in version 1}|{Incorrect in version 2}|{Correct in version 3}@}
	
	\choice{Choice 4 - Incorrect in all versions}
	
	\choice{[2]Choice 5 - @{Incorrect in version 1}|{Correct in version 2}|{Incorrect in version 3}@}
	
\end{question}


\begin{question}
	
	Consider the following statements:
	
	\begin{enumerate}[I]
		\item Latex is @{good}|{bad} for structured documents@
		
		\item Working with R and latex is a @{good}|{bad}@ choice for writing up and marking tests
		
		\item We @{love}|{hate}@ R
		
	\end{enumerate}
	
	Which statements are true?
	
	\choice{I e II}
	\choice{[1] I, II e III}
	\choice{Only III}
	\choice{I e III}
	\choice{[2] None of the other options}
	
\end{question}

\end{multiplechoice}



\end{document}
